\documentclass[authoryear, review, 12pt]{elsarticle}
\usepackage{amssymb,amsmath,amsfonts,graphicx}
\bibliographystyle{elsarticle-harv}
%\usepackage{natbib} % already loaded by elsarticle
%\biboptions{sort&compress} % For natbib
%\usepackage{hyperref} % already loaded by elsarticle
%\hypersetup{breaklinks=true, pdfborder={0 0 0}}

\usepackage[hyphens]{url}


\usepackage{lineno}
\linenumbers

\usepackage{subfigure}
%\usepackage[nomarkers]{endfloat}
\renewcommand{\listoffigures}{} % but suppress these lists
\renewcommand{\listoftables}{} % suppress these lists



%% Redefines the elsarticle footer
\makeatletter
\def\ps@pprintTitle{%
\let\@oddhead\@empty
\let\@evenhead\@empty
\def\@oddfoot{\it \hfill\today}%
\let\@evenfoot\@oddfoot}
\makeatother
% A modified page layout
\textwidth 6.75in
\oddsidemargin -0.15in
\evensidemargin -0.15in
\textheight 9in
\topmargin -0.5in


\usepackage{microtype}
\usepackage{fancyhdr}
\pagestyle{fancy}
\pagenumbering{arabic}

\usepackage[pdftex, usenames]{color}
\definecolor{darkblue}{rgb}{0,0,0.5}
\definecolor{darkgreen}{rgb}{0,0.5,0}
\newcommand{\cdb}[1]{{\it \color{darkgreen} #1}}
\newcommand{\pa}[1]{{\it \color{darkblue} #1}}

\begin{document}
\begin{frontmatter}
\title{Policy Costs}
\author[cpb]{C. Boettiger\corref{cor1}}
\ead{cboettig@ucdavis.edu}
\author[melbourne]{M. Bode}
\author[esp]{J. Sanchirico}
\author[utk-econ]{J. LaRiviera}
\author[cpb,esp]{A. Hastings}
\author[utk-eeb]{P. Armsworth}
\cortext[cor1]{Corresponding author}
\address[cpb]{Center for Population Biology, University of California, Davis, California 95616, USA}
\address[esp]{Department of Environmental Science \& Policy, University of California, Davis, CA 95616, USA}
\address[utk-econ]{Department of Economics, University of Tennessee, Knoxville, TN 37996, USA}
\address[utk-eeb]{Department of Ecology and Evolutionary Biology, University of Tennessee, Knoxville, TN 37996, USA}
\address[melbourne]{Australian Research Council Centre of Excellence for Environmental Decisions, University of Melbourne, School of Botany,
Parkville, Melbourne, VIC 3010, Australia}

\begin{abstract}
\cdb{Carl's comments \& questions added in green}
\end{abstract}
\begin{keyword}
  optimal control \sep management \sep stochastic dynamic programming \sep fisheries 
\end{keyword}

\end{frontmatter}

\section{Introduction}


\emph{Could use input from those most familiar with this literature to flush this out more.}

\begin{itemize}
  \item Ecosystem management frequently set in terms of policy of quotas. 
  \item Policy relatively static and costly to change
  \item Meanwhile, the natural world is variable
  \item Optimal solutions typically track these shocks, resulting in impractical management recommendations in face of policy costs
\end{itemize}

(See Figure~\ref{fig:1} which motivates the importance of considering policy costs by comparing a theoretically optimal policy over time to a typical real world policy over time).  

\begin{figure}[ht]
  \begin{center}
    \subfigure[Optimal Policy]{\includegraphics[width=.5\textwidth]{figure/optimal_harvest}\label{fig:1a}}
    \subfigure[Typical Policy]{\includegraphics[width=.5\textwidth]{figure/typical_policy}\label{fig:1b}}
  \end{center}
  \caption{\ref{fig:1a} shows a typical solution found by an optimal
  control solution by stochastic dynamic programming for the optimal catch
  quota in a managed fishery.  Note that the solution tends to track the
  noise, resulting in a highly variable policy.  \ref{fig:1b} shows the
  actual quotas set over the past decade for Bluefin tuna, which show
  great intertia to change despite considerable movement in the stock
  dynamics.  \cdb{Conceptual picture illustrating the need to account
  for costs to changing policy. Current images are just placeholders.
  B should be replaced with real data, perhaps tuna quotas from Paul.
  Should stock dynamics actually be shown?}}
  \label{fig:1}
\end{figure}

%Optimal management of ecosystems and natural resources often involves
%frequent adjustment of a control variable in response to the observed
%state of a system. When this control is set in a policy-making process,
%such as determining a fishing quota, it may be more costly to change
%the policy in response to new information than to continue with the
%status quo ~\citep{Bohm1974, Xepapadeas1992}.  A management strategy that
%is optimal when adjustment costs are free may be costly and inefficient
%when these costs are present.  Meanwhile, this inertia to policy change
%introduces a two-fold cost to optimal management solutions that account
%for it: once in direct costs of changing policy, and another in the
%differences from optimality.

\subsection{Types of real world policy costs}

\begin{itemize}
  \item A discussion of capital adjustment costs, smoothing, \emph{e.g.} \citet{Singh2006}.
  \item Historical context, \emph{e.g.} discussion of \citep{Bohm1974, Reed1979, Xepapadeas1992}.  
\end{itemize}
\subsection{Stochastic fisheries model}
\begin{itemize}
\item Historical context of the Reed (1979) model.  
\item Discuss the importance and relevance of stochastic, discrete time models, which frequently yield highly variable solutions since the policy tracks the shock.  
\end{itemize}


\section{Methods}

\subsection{Model setup}

\begin{itemize}
  \item   Fish population dynamics / state equation.  We will assume Beverton-Holt dynamics,
    \begin{equation} 
      X_{t+1} = Z_t \frac{A X_t}{1 + B X_t}, 
    \end{equation}
where $Z_t$ gives the stochastic shocks.  $Z_t$ may be distributed log-normally \footnote{though this violates the self-sustaining property of~\citet{Reed1979}, such that the optimal constant-escapement level $S$ in the stochastic model is greater than the optimal escapement in the deterministic scenario.}

\item   Fishing profit function 
  \begin{equation} 
    \Pi_0(x,h) = p h - \left( c_0  + c_1 \frac{h}{x} \right) \frac{h}{x} \label{profit}
  \end{equation}
  where $h$ is the harvest level, $x$ the stock size, $h/x$ represents
  fishing effort, $p$ the price per unit harvest.  The coefficient $c_1$
  introduces a quadratic cost to effort, a typical way to introduce
  smoothing~\citep[\emph{e.g.}][]{Singh2006}.  For simplicity, we will consider $c_0 =
  c_1 = 0$.

\item
Policy cost function: L1, L2, fixed fee, asymmetric, are introduced
as modifications to the cost function that depend on the action
taken in the previous time-step.  (This makes the previous action part
of the state space).   All cost functions are characterized in terms
of the coupling coefficient $c_2$.  (Because $c_2$ takes different units
and strength of interaction under the different functional forms, it is
necessary to calibrate the choice of this coefficient such that these
penalties can be compared directly, as described in the next section.)

  \begin{align} 
    \Pi_{L_1}(x_t,h_t, h_{t-1}) &= \Pi_0 + c_2 \operatorname{abs}\left( h_t - h_{t-1} \right) \label{L1} \\
    \Pi_{L_2}(x_t,h_t, h_{t-1}) &= \Pi_0 + c_2 \left( h_t - h_{t-1} \right)^2 \label{L2} \\
    \Pi_{\textrm{fixed}}(x_t,h_t, h_{t-1}) &= \Pi_0 + c_2 \mathbb{I}(h_t, h_{t-1})  \label{fixed_fee} \\
    \Pi_{\textrm{asym}}(x_t,h_t, h_{t-1}) &= \Pi_0 + c_2 \operatorname{max}\left( h_t - h_{t-1}, 0 \right) \label{asym}
  \end{align}
  Where $\mathbb{I}(a,b) = 0$ for $a \neq b$ and  $\mathbb{I}(a,b) = 1$ for $a = b$.  

\item   Economic discounting, boundary conditions, constraints, Bellman equation, SDP solution method on finite time horizon.  
\item   Choice of model parameters 
\end{itemize}


\subsection{Apples to apples comparisons}

% From Jake: 

The fact that there is no uniform metric for comparing different functional forms of the penalty function complicates identifying the implications of various penalty functions on optimal management.  Put another way, it is unclear what parameter values should be used for each functional form in order to compare the relative effect of each penalty function on optimal management.  
To address this issue, consider the following rule for comparing functional forms inspired by Bovenberg, Goulder and Gurnery (2005) and Bovenberg, Goulder and Jacobsen (2008).  By definition, adding a constraint, in the form of a penalty function, to the social planner’s objective function will reduce the NPV of the resource in addition to affecting the optimal management policy.  For example, each individual penalty function equation (3)-(6) is multiplied by a constant C2.  As the magnitude of the penalty function increases (e.g., C2 increases), the NPV value of the resource decreases.  Importantly, though, the rate at which the NPV is affected by a given increase in magnitude of the penalty function is different for different penalty functions.  One way to compare the relative impact of penalty functions on optimal management is to choose parameters for the penalty function that imply the same level of NPV conditional on optimal management. 

Figure 2 shows this comparison criterion graphically.  Each curve plots the percentage change in NPV relative to the unconstrained problem for a given penalty function over different magnitudes of that penalty function, C2, given optimal management.  There are two important features in Figure 2.  First, the optimal management policy is not constant across penalty function.  This will be discussed in detail below.  Second, the rate of change in NPV as a function of the magnitude of the penalty function varies both within and across policies.  For example, the existence of a relatively small L2 penalty function dramatically affects the resource’s value initially but there is little marginal effect on NPV for relatively large L2 penalty functions.  Conversely, the rate of change in NPV for the fixed penalty function is roughly constant over all ranges of penalty function magnitudes.  

To compare the impact of penalty functions on optimal management, we select penalty magnitudes that make the resource worth 80\% of its unconstrained value.   The dashed vertical lines in Figure 2 map the needed penalty function magnitude, C2, to each penalty function such that the resource is worth 80\% of its unconstrained value when optimally managed.  We choose 80\% somewhat arbitrarily but the method is applicable to any NPV level and the qualitative effects on management are robust for other levels.  

Bovenberg, AL, Goulder LH, and Gurney DJ (2005).  “Efficiency Costs of Meeting Industry-Distributional  Constraints Under Environmental Permits and Taxes” RAND Journal of Economics. 36(4): 951-  971.

Bovenberg, AL, Goulder LH, and Jacobsen MR (2008).  “Costs of Alternative Policy Instruments in the    Presence of Industry Compensation Requirements” Journal of Public Economics. 92: 1236-1253.

%% end-Jake



We compare the impact of the different functional forms for the penalty
function at a value of the penalty scaling parameter in each model that
induces an equivalent net present value to the stock when optimally
managed under this penalty. This value represents the expected net
present value before the costs of policy adjustment are paid, Figure~\ref{fig:apples}

  \begin{figure}
    \begin{center}
\includegraphics[width=.8\textwidth]{figure/npv0.png}
\caption{\textbf{Net present value by functional form of penalty.} Horizontal axis shows the coefficient $c_2$ governing the magnitude of the policy cost, while vertical axis shows fraction of the stock value dissipated by the policy cost.  The horizontal line indicates a value of the stock that is 80\% of the adjustment-free value. Selecting the coefficient $c_2$ corresponding to this value in each functional form allows us to make consistent comparisons across the different functional forms of policy costs.  }
\label{fig:apples}
\end{center}
\end{figure}


\section{Results}
\subsection{Characterize and compare functional forms of costs}

Figure~\ref{fig:shapes} shows the impact the adjustment cost has on the optimal policy.  The $L_1$-norm costs tend to avoid small adjustments to the policy, resulting in long periods of constant policy followed by sudden bursts of adjustment.  In contrast, the $L_2$-norm tracks all of the changes made by the cost-free policy, but with smaller magnitude.  The fixed fee only makes large adjustments much like the $L_1$, as expected where the adjustment magnitude does not impact the cost.  Asymmetric ($L_1$) adjustment costs do not appear to show more directional bias than any of the symmetric costs.  Note that while the cost-free optimal strategy corresponds to a constant-escapement policy as derived in~\citet{Reed1979}, any of the policy-cost scenarios deviate from such a policy.  
    \cdb{
    \begin{itemize}
      \item All of these should use the same sequence of shocks.  
      \item Don't plot all four curves on the same graph.  
      \item Omit stock dynamics?
      \item Consider not violating the self-sustaining property, which should also guarantee that fishing never stops.   
    \end{itemize}
    }

\cdb{ In terms of a policy set on escapement instead of harvest, our solutions are actually more volatile than~\citet{Reed1979}.  Maybe this leads to a discussion of what the control policy should be defined in terms of (i.e. catch vs escapement)?  Surely the simplicity of the constant-escapement rule has led to it's popularity in some part due because that simplicity avoids adjustment costs?}

  \begin{figure}
    \begin{center}
      \subfigure[$L_1$]{\includegraphics[width=.45\textwidth]{figure/7258516896_5c89f034d5_o.png}\label{fig:L1}}
      \subfigure[$L_2$]{\includegraphics[width=.45\textwidth]{figure/7258563112_2f5f9ffecd_o.png}\label{fig:L2}}

      \subfigure[Fixed costs]{\includegraphics[width=.45\textwidth]{figure/7258506664_d6235e5f8e_o.png}\label{fig:fixed}}
      \subfigure[Asymmetric costs]{\includegraphics[width=.45\textwidth]{figure/7258432026_d6f8179f54_o.png}\label{fig:asym}}
  \end{center}
  \caption{Individual realizations of management under the different functional forms of adjustment costs.  \textbf{Green} lines show the cost-free optimal harvest levels, \textbf{black} lines show the corresponding stock dynamics under that management.  \textbf{Purple} lines show the optimal harvest level under the specified adjustment cost, under the same sequence of stochastic shocks as in the cost-free policy.  \textbf{Blue lines} show the resulting stock dynamics.
  } \label{fig:shapes}
\end{figure}



\subsection{Responses to increasing costs}

The comparisons shown in Figure~\ref{fig:shapes} are made at a fixed magnitude of the policy cost equal to 80\% of the net present value derived from the cost-free optimal strategy.  The different functional forms may also respond differently to an increasing magnitude of the adjustment cost.  To quantify the extent to which the policy cost decreases the volatility of the proposed policy, we calculate the variance in the fishing quota over time under each policy.  While a decrease in variance is typical of smoothing penalties and the $L_2$-norm, the patterns in Figure~\ref{fig:shapes} suggest that other functional forms still involve the same magnitude of variation, but decrease the frequency of adjustments.  To capture this behavior we calculate the autocorrelation of the policy as the adjustment cost increases.  These curves van be seen in Figure~\ref{fig:summarystats}.  


\cdb{Perhaps not clear what is comparable here and what isn't.  Certainly we cannot compare the curves at the same value of $c_2$, as discussed in our apples to apples section.  Not clear that we can even compare the shape of the curves though -- for instance, given the units of $c_2$ for the $L_2$ norm are different than the other norms.  Perhaps the horizontal axis should be converted into percent of net present value of the adjustment free costs first?}

\cdb{Add additional plot: Is value a linear function of policy cost, or is there a sharp transition? }
%Must be nonlinear somewhere when adjustment costs exceed profits?




\begin{figure}
  \begin{center}
    \subfigure[Variance]{\includegraphics[width=.45\textwidth]{figure/6850042286_ef81b74acc_o.png}\label{fig:var}}
    \subfigure[Autocorrelation]{\includegraphics[width=.45\textwidth]{figure/6996165783_41c9894bdb_o.png}\label{fig:acor}}
    \caption{Variance and autocorrelation as a function of the magnitude of penalty, by penalty function}\label{fig:summarystats}
  \end{center}
\end{figure}


\subsection{Consequences of ignoring policy adjustment costs}

\pa{
   
  \textbf{Option 1.}

  We don’t include do a section 3.3. I’m not convinced the paper needs one. Some of this I think is a tension between a focus in the paper on :

  A.      ``What are the effects of including policy adjustment costs on optimal policy recommendations at all?''

  Vs.

  B.      ``Given that there will be costs to adjusting policies but we don’t know just what form these will take, what are the consequences of different cost formulations''



  If we focus the paper on question B there is less need for a section 3.3 at least as Carl has that section currently entitled (``Consequences of ignoring policy adjustment costs'' – which is not really a statement about functional forms).
   

  \textbf{Option 2.}

  We include a version of the results Carl currently shows in his Fig 5 and 6 for the L1 penalty, but we include it also for the other penalty function forms – e.g. using Carl’s suggestion of a box-whisker plot.
   
  Reminder - The two figures compare the performance of the optimal policy when accounting for the policy adjustment costs (maximizing $\Pi_{L_1}$) with what the optimal policy recommendation one would arrive if ignoring these costs (maximizing $\Pi_0$) WHEN scoring the performance of the two control recommendations against $\Pi_{L_1}$.
   
  Fig. 5 shows the breakdown into how the gross yield (ph) and adjustment cost ($c_2 abs(h_t-h_{t-1})$) separately. The policy ignoring costs of policy adjustment gets higher gross yields by tracking the variations more closely but also incurs (even) higher policy adjustment costs. Fig 6 combines the two bits to show the effect on net present value.
   
  This option would let one answer:

  i)                    What is the consequence of ignoring policy adjustment costs when they are present?

  ii)                   It also answers Jim’s suggested reversal of that question, I think, namely: what is the consequence of assuming adjustment costs are present when in fact they are absent? (I think this is just the difference between the gross yields / the turquoise and red shaded areas in Fig 5).

  iii)                 And then how do these things vary for each of the different functional forms?

   
  \textbf{Option 3}


  We populate a table that organizes columns by the ``Actual functional form of adjustment costs'' and the rows into ``Assumed functional form of adjustment costs.'' The types of functional form in both columns and rows would then run ``None, $L_1$, $L_2$ etc.''
   
  IF we denote the optimal control for objective function $NPV_i$ as being $h^*_i$, then I would populate the Table with the proportions
  $NPV_i(h^*_j) / NPV_i(h^*_i)$
  (i.e. what proportion of the available NPV you still get if assuming the wrong functional form for the adjustment costs).
   
  The inclusion of the comparison against None means that all of what was being talked about for Option 2 would be captured. In addition it would provide a robustness check on a policy recommendation if uncertain of either the presence of policy adjustment costs or functional form that they should adopt. Specifically, by comparing across rows one could ascertain whether assuming one choice of functional form was likely to lead to large efficiency costs should other assumptions about adjustment costs prove correct. Similarly one could compare across columns to see if one form were particularly problematic in reality.
   
  I don’t know if those are clear – you need to get mired into the document once again a bit to parse it out. I’ll happily discuss with anyone in a bid to shed more light.
   
  Discussion of these options and alternative suggestions invited.
}



\begin{itemize}
  \item Distribution of profits and costs under a policy-cost scenario vs profits and costs under Reed solution, by penalty type. Figure~\ref{fig:profits_costs}.
  \item Distribution of net present value under the policy-cost scenario vs the Reed solution, Figure~\ref{fig:npv_dist}.  
\end{itemize}

\begin{figure}
  \begin{center}
    \subfigure[L1]{\includegraphics[width=.45\textwidth]{figure/7751440758_99645d6e5a_o.png}\label{fig:L1cost}}
    \caption{Ignoring the cost of policy adjustment results in greater profits from fishing, but even greater costs for adjusting the harvest quotas each year.  Costs and profits are shown for ignoring the adjustment costs (Reed solution, purple and blue) and accounting for the adjustment (green and red).  \cdb{Though only L1 shown for the moment, would probably show all four types.  Possibly a table or box-and-whiskers style plot would be better?}}\label{fig:profits_costs}
  \end{center}
\end{figure}

\begin{figure}
  \begin{center}
    \subfigure[L1]{\includegraphics[width=.45\textwidth]{figure/7751440918_3b54ff2397_o.png}\label{fig:L1npv_dist}}
    \caption{\textbf{The impact of ignoring policy costs}.  Distribution of net present value realized over 100 simulations when adjustment costs are ignored (blue), compared to when adjustment costs are optimally accounted for (red). \cdb{Though only L1 shown for the moment, would probably show all four types.  Possibly a table or box-and-whiskers style plot would be better?}}\label{fig:npv_dist}
  \end{center}
\end{figure}




Add two-by-two table of net present value accounting for:  ``Adjustment cost in policy calculation'' by ``adjustment cost in reality''?

\cdb{\textbf{Further things to look at?}
\begin{itemize}
  \item How are costs partitioned between policy adjustments and deviation from optimality? 
  \item Risk sensitivity: does policy inertia increase or decrease extinction risk? Does accounting for the policy cost increase or decrease extinction risk?  (only in cases where extinction is possible: \emph{e.g.} in large variance, outside of Reed's self-sustaining range.  Or Allee model)
\end{itemize}
}


\section{Discussion}
\cdb{Some possible things to discuss: could add others, could remove some.}
\begin{itemize}
\item   Comparison of the optimal policy solution to the Reed model
\item  Comparison between the different functional forms -- smoothing (L2)
  non-smoothing (L1), and de-stablizing (fixed).
\item   Induced costs (relative to free adjustment) vs direct costs (paid for adjusting) 
\item   Impact of symmetric vs asymmetric costs
\item   Contrast steady-state results to dynamic solutions under stochastic shocks.
\end{itemize}

\textbf{Key conclusions:}
\begin{enumerate}
  \item \textbf{Lesson for modelers}: Ignoring the reality of policy costs leads to less effective management (decreased value)  (increased extinction risk?)
  \item \textbf{Lesson for policy makers}: Decreasing adjustment costs not only saves direct costs, but permits higher-value solutions.
\end{enumerate}


\section*{References}
\bibliography{outline_refs.bib}

\end{document}
